\documentclass[12pt]{article}
\usepackage{amssymb}
\usepackage{mathrsfs}
\addtolength{\topmargin}{-.5in} \addtolength{\textheight}{1in}
\addtolength{\oddsidemargin}{-.6in}
\addtolength{\evensidemargin}{-.6in} \addtolength{\textwidth}{1.2in}
\usepackage{latexsym,amsmath,amssymb,amsfonts,epsfig,graphicx,cite,psfrag,ctex}
\usepackage{eepic,color,colordvi,amscd}
\usepackage{xeCJK}
\setCJKmainfont{Source Han Serif CN}

\def\qed{\hfill \rule{4pt}{7pt}}
\begin{document}
\begin{center}
            \large{代数学第二次作业}
\end{center}
\section{课堂练习}
\begin{enumerate}
    \item 设$M$为$S,T$内直和, 则$M/S\tilde{\rightarrow}T$, $M/T\tilde{\rightarrow}S$.
    \item $q,i,j,\pi,A,B$如课堂讲述. 请验证:$q$为同态, $q\circ i=\mathrm{Id}_A,\ q\circ j=0,\ \mathrm{Id}_B=j\circ\pi+i\circ q$.
\end{enumerate}
\section{课本习题}
\begin{enumerate}
    \item 设$\{M_i\}_{i\in I}$为一族$R$-模, 对每个$i$, $N_i\subset M_i$为子模, 证明:$$(\bigoplus_{i\in I}M_i)/(\bigoplus_{i\in I}N_i)\tilde{\rightarrow}\bigoplus_{i\in I}(M_i/N_i)$$
    \item 设$0\rightarrow A\rightarrow B\rightarrow C \rightarrow 0$为模的短正合列, $M$为任意模, 证明: 存在短正合列$$0\rightarrow A\oplus M\rightarrow B\oplus M\rightarrow C \rightarrow 0$$及$$0\rightarrow A\rightarrow B\oplus M\rightarrow C\oplus M \rightarrow 0$$
    \item 设$V_i(0\leq i\leq n)$是有限维$k$-线性空间, $$0\rightarrow V_0\rightarrow V_1\rightarrow \dots \rightarrow V_n \rightarrow 0$$为$k$-线性空间正合列. 证明:$$\sum_{i=0}^n (-1)^i \mathrm{dim}_k V_i=0$$
    \item 如$A\xrightarrow{f}B\xrightarrow[]{}C\xrightarrow{h}D$为正合列, 证明: $f$为满射当且仅当$h$为单射.
    \item 在$R$-模范畴中证明: $0\rightarrow M^{'}\rightarrow M \rightarrow M^{''}$为正合列当且仅当对任意$R$-模$N$, $0\rightarrow \mathrm{Hom}(N,M^{'})\rightarrow \mathrm{Hom}(N,M) \rightarrow \mathrm{Hom}(N,M^{''})$是正合列.
    \end{enumerate}
  

\end{document}